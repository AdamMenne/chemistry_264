\documentclass[a4paper, british]{article}

\usepackage[utf8]{inputenc}
\usepackage[T1]{fontenc}
\usepackage{babel}
% \usepackage[margin=2.5cm,a4paper]{geometry}
% \usepackage[skip=1em]{parskip}
\usepackage{lmodern} 
\usepackage{microtype}
% \usepackage{xcolor}
\usepackage{graphicx}
\graphicspath{ {./figures/} }
\usepackage{float}
\usepackage{enumitem}
\usepackage{adjustbox} % rescale - useful for Dia exported TeX
\usepackage{tikz}
% \usepackage{pgfplots}
\usepackage{booktabs} %tables no vertical lines
% \usepackage{array}
% \usepackage{authblk}
% \usepackage{fancyhdr} %headers and footers
% \usepackage{titlesec}
% \usepackage{tcolorbox} % framed text boxes
% \usepackage{mathtools, amssymb, amsthm}
% \usepackage{gensymb}
\usepackage{chemformula} % chemical formulae
\usepackage{chemfig} % molecular figures
\usepackage{siunitx}
\usepackage{csquotes}
\usepackage[titletoc, title]{appendix}
% \usepackage{lettrine} % initials

\usepackage[
pdfauthor={Adam Menne},
pdftitle={Chemistry 264 - Practical 7},
pdfsubject={},
pdfkeywords={}]{hyperref}

\usepackage[noabbrev]{cleveref}

\usepackage[
backend=biber,
style=numeric,
sorting=none,
doi=true,
isbn=false
]{biblatex}
\addbibresource{citations.bib}

\setlength{\parskip}{1em}
\setlength{\parindent}{0em}
\linespread{1.3}

\title{Chemistry 264\\ Practical 7}
\date{Last edited on \today}
\author{Adam Menne\\ Stellenbosch University}

\begin{document}

\maketitle

\begin{abstract}
\noindent
In this practical gas chromatography was used to analyse a solution of hydrocarbons. From this the effect of various parameters related to gas chromatography were investigated.
\end{abstract}

\tableofcontents

\newpage

\section{Experiment}

A solution containing heptane through decane in a hexane solvent was analysed using gas chromatography, four runs were done in total, three isothermally at 60\textcelsius{}, and one using a temperature program. The results were used to investigate the reproducibility of manual injection in gas chromatography, and the effect of film thickness of the column used.


\section{Calculations}

\subsection*{1.}

The retention time of nonane had a RSD of 0.2599\%, while the peak area was 133.1\%. The high RSD of peak area shows that the volume of sample introduced to the column was not consistent over the three runs.

\subsection*{2.}

The values for the retention factor of heptane through decane found in \cref*{tab:k} for a column with film thickness of \(0.25\mu m\) are approximately half that of those for a column with film thickness of \(0.5\mu m\). This is due to the decreased exposed surface area of the stationary phase that comes with increased thickness, leading to increased retention times.

\begin{table}[H]
    \centering
    \caption{}
    \vspace*{2mm}
    \label{tab:k}
    \begin{tabular}{cc}
    \toprule
    Analyte & Retention factor \\ \midrule
    \(C_7\) & 0.1282 \\
    \(C_8\) & 0.4162 \\
    \(C_9\) & 1.013 \\
    \(C_{10}\) & 2.244 \\ \bottomrule
    \end{tabular}
\end{table}

\subsection*{3.}

For a column with a length of \(30 m \), a \( mm\)  internal
diameter, and a film coating of \(0.25 \mu m\) thickness, the distribution coefficient of octane and nonane were 104.0 and 253.3 respectively.

\subsection*{4.}

The selectivity factor \(\alpha\) between octane and nonane was 2.434

\subsection*{5.}

The peak width of octane at half-height was 0.01938 minutes. With a concurrent separation efficiency of 108000.

\subsection*{6.}

The height equivalent of a theoretical plate was 0.0002777

\subsection*{7.}

The coating efficiency was 90.02\%

\subsection*{8.}

Yes, it will have a short retention time due to being extremely polar, and the PDMS stationary phase not being polar.

\subsection*{9.}

It allows for the length of the analysis to be decreased, as the higher the temperature the faster the mobile phase moves through the column. As such a good temperature programmed run will allow the early eluting analytes to separate adequately while decreasing the time between later eluting analytes to pass through the column.

\subsection*{10.}

\begin{enumerate}[label={\alph*)}]
    \item The column will be saturated and analytes will elute simultaneously, leading to peaks overlapping.
    \item Significantly increased retention time.
    \item Increased overlapping
    \item Initial peaks will be closer together and later will be further apart.
    \item For an isothermal run, retention time should have a fairly linear proportionality with boiling point, however as this is a temperature programmed run, deducing the relative boiling points of the analytes is not possible without the parameters used in the temperature program.
\end{enumerate}

A static export of the notebook containing all analysis and figures is availible at \url{https://adammenne.github.io/chemistry_264/practical_7/notebook.html}. With full source code availble at \url{https://github.com/AdamMenne/chemistry_264/tree/master/practical_7}


\end{document}